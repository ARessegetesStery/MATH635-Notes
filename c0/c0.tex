\documentclass{article}
\usepackage{../refalg}

\begin{document}
\Makepagesectionhead{MATH 635}{Basic Definitions and Results}{ARessegetes Stery}

\tableofcontents
\newpage

\def\open{\mathcal{O}}

Here lists some definitions that are assumed to be familiar to students in the lecture. They are listed for an acquaintance with the basic definitions, and for lookup. For a gentler introduction to basic differential geometry, please refer to \href{https://github.com/ARessegetesStery/MATH437-Notes}{this note}.

\section{Differential Form}

\textstart
This part seeks to introduce the differential between tangent bundles of manifolds, and its dual operation. To formalize this, some preliminaries from algebra need to be introduced first:

\begin{definition}[Tensor Algebra]
    Let $R$ be a ring, with $M$ an $R$-module. Then the \textbf{tensor algebra} on $M$ is defined to be
    \[
        T(M) := \bigoplus_{i \geq 0} T^i(M) \qquad\text{where } T^i(M) := \underbrace{M \tensor_R \cdots \tensor_R M}_{i \text{ times}}
    \]
\end{definition}

\begin{definition}[Exterior Algebra]
    Let $T(M)$ be the tensor algebra on $M$. Then the \textbf{exterior algebra} on $M$ is defined as
    \[
        \Lambda(M) := T(M) / I , \qquad \text{where } I := \left\{ \inner{x \tensor x} \mid x \in M \right\}
    \]
    The equivalence classes in $\Lambda(M)$ is denoted by the \underline{wedge prduct}, i.e.
    \[
        a \wedge b := [a \tensor b] = (a \tensor b) \mod I
    \]
\end{definition}

\begin{remark}
    By the quotient applied to the tensor algebra, it is alternative. Specifically, for all $x, y \in M$, $x \wedge x = 0$; and $x \wedge y = -y \wedge x$.
\end{remark}

\begin{remark}
    Similar to the tensor algebra, the exterior algebra can be decomposed as
    \[
        \Lambda(M) = \bigoplus_{i \geq 0} \Lambda^i(M), \qquad \text{where } \Lambda^i(M) := T^i(M) / \{ \inner{x \tensor x} \subset T^i M \mid x \in M \}
    \]
\end{remark}

\begin{definition}[Tangent Space]
    There are two equivalent definitions of tangent spaces, via specifying its elements \textbf{tangent vectors}. Let $M$ be a smooth $n$-manifold, with $(U, \varphi)$ a local chart of it. Then the \textbf{tangent space} of $M$ at $p$ is defined as:
    \begin{itemize}
        \item \emph{Tangent vectors as equivalence class of curves.} Let $\gamma: R \supset I \to M$ be a curve on $M$. Define the equivalence class of such curves as
        \[
            \gamma_1 \sim \gamma_2 \quad \iff \quad (\varphi \circ \gamma_1)'(0) = (\varphi \circ \gamma_2)'(0)
        \]
        The \textbf{tangent vector}s are then equivalence classes of curves on $M$. 
        
        Considering the map $(\varphi \circ -)'(0) : T_p M \to \R^n$, which by the definition of the equivalence relation is a bijection. The vector space structure is then induced by the vector space structure in $\R^n$.
        \item \emph{Tangent vectors as derivation.} Denote $f: M \to \R$ as $\smooth(M)$ if $f$ is smooth, i.e. for chart of $M$ $(U, \varphi)$ and $p \in U$, then in the neighborhood of $p$, $f \circ \varphi$ is smooth as maps from $\R^n$ to itself. Then a \textbf{tangent vector} is a map $\partial: \smooth(M) \to \R$ s.t.
        \[
            \forall f, g \in \smooth(M), \qquad \partial (f \cdot g) = \partial(f) \cdot g + f \cdot \partial(g)
        \]
        i.e. the \underline{Leibniz Identity}. The multiplication $\cdot$ is the multiplication in $R$ after applying the map. The vector spae structure is specified via requiring the functions to be linear, i.e. define
        \[
            \partial(\lambda f + \mu g) = \lambda \cdot \partial(f) + \mu \cdot \partial(g), \qquad \forall f, g \in \smooth(M), \lambda,\ \mu \in \R
        \]
    \end{itemize}
\end{definition}

\begin{definition}[Cotangent Space]
    The \textbf{cotangent space}, as the dual of tangent space, is defined similarly in two equivalent ways:
    \begin{itemize}
        \item \emph{Explicitly as dual of tangent space.} For $p \in M$ which is a smooth $n$-manifold, the \textbf{cotangent vectors} $\dual{T_p} M := \dual{(T_p M)}$ are \emph{linear} functions $T_p M \to \R$ (under the context of real manifolds). The cotangent space is generated by the cotangent vectors, with the vector space structure induced by the real vector space.
        \item \emph{Equivalence class of smooth functions.} Let $f, g \in \smooth(M)$. Then for $p \in M$, 
        \[
            f \sim g \quad \iff \quad \restr{D(f - g)}{p} = 0
        \]
        The cotangent space consists of equivalence class of such smooth functions. The vector space structure is the same as that of real functions.
    \end{itemize}
\end{definition}

\begin{remark}
    In the second definition for cotangent space, this is exactly the dual of the second definition of tangent spaces. This can be seen by the fact that if two functions are in the same equivalence, the derivation on them should be identical.
\end{remark}

\begin{definition}[Differential Form]
    Let $M$ be a manifold, with $U$ an open subset of $M$. Then a \textbf{$\bm{k}$-form} on $U$ is a map which associates every $p \in U$ an element in the exterior algebra $\omega_p \in \Lambda^k(\dual{T_p} M)$
\end{definition}

\begin{remark}
    In particular, smooth functions are 0-forms, and cotangent vectors are 1-forms.
\end{remark}

\begin{remark}
    By the universal property of tensor product, for $\wedge^k(M)$ we have the isomorphism
    \[
        \dual{(w_1 \wedge \dots \wedge w_k)} \simeq \dual{w_1} \wedge \dots \dual{w_k}
    \]
    i.e. there is an induced isomorphism between tensor of 1-forms and $k$-forms, via taking the tensor product of the 1-forms.
\end{remark}

\begin{definition}[Pushforward, Pullback]
    Let $F: M \to N$ be a smooth map between manifolds, and $p \in M$. Then
    \begin{itemize}
        \item The \textbf{pushforward (differential)} of $F$ at $p$ is defined as
        \[
            \d_p F: T_p M \to T_{F(p)} N, \qquad \d_p F (\partial) (\varphi) = \partial(\varphi \circ F)
        \]
        for $\partial \in T_p M \simeq (\smooth(M) \to \R), \varphi \in \smooth(M)$.
        \item The \textbf{pullback} of $F$ at $p$ is defined on the \underline{covector field}, or \underline{linear forms}, defined as 
        \[
            \dual{(\d_p F)}: \dual{T_{F(p)}}N \to \dual{T_p} M, \qquad \dual{(\d_p F)}(f)(\partial) = f(\d_p F(\partial))
        \]
        where $f \in \dual{T_{F(p)}} N, \partial \in T_p M$.
    \end{itemize}
\end{definition}

\begin{definition}[Immersion, Submersion]
    The following is a pair of dual notions which specifies the differential pushforward of a smooth map $f: M \to N$:
    \begin{itemize}
        \item $f$ is an \textbf{immersion} if for all $p \in M$, $d_p f : T_p  M \to T_{f          (p)} N$ is injective.
        \item $f$ is a \textbf{submersion} if for all $p \in M$, $d_p f : T_p  M \to T_{f(p)} N$ is surjective.
    \end{itemize}
\end{definition}

\section{Exterior and Interior Derivative}

\textstart
Under the context of differential geometry it is then interesting to figure out how the differential forms can be ``differentiated'' in a way that is compatible with the concept of derivatives of real-valued functions (i.e. $0$-forms). First we need some tools to consider how the differential forms act upon wedge products (the generalization of multiplication in the context of exterior algebra):

\begin{remark}\label{rmk: wedge of differential forms}
    Formally, the alternating product (wedge product) is viewed as an endomorphism on the tensor algebra:
    \[
        \operatorname{Alt}: T(M) \to T(M) \qquad (v_1 \tensor \cdots \tensor v_k) \mapsto \frac{1}{k!} \sum_{\sigma \in S_k} \operatorname{sgn}(\sigma) v_{\sigma(1)} \tensor \cdots \tensor v_{\sigma(k)}
    \]
    Therefore, given $\alpha$ a $p$-form, and $\beta$ a $q$-form, the wedge product of $\alpha$ and $\beta$ is (by the same averaging process)
    \[
        (\alpha\wedge\beta)(w_1, \dots, w_{p+q}) = \sum_{\sigma \in S(p, q)} \operatorname{sgn}(\sigma) \alpha(w_{\sigma(1)}, \dots, w_{\sigma(p)}) \beta(w_{\sigma(p+1)}, \dots, w_{\sigma(p+q)})
    \]
    where $\sigma \in S(p, q)$ denotes the permutation which preserves the relative order in first $p$ and last $q$ elements, respectively. 
\end{remark}

\begin{definition}[Graded Ring]
    A ring $R$ is a \textbf{graded ring} is there exists subgroups $R_0, \dots, R_n, \dots$ of $R$ seen as an additive subgroup s.t. we have the decomposition of $R$
    \[
        R = \bigoplus_{n = 0} ^ {\infty} R_n
    \]
    and further for all $m, n \in \Z_{\geq 0}$, $R_m R_n \subseteq R_{m + n}$. 

    An algebra is \underline{graded} if it is graded when viewed as a ring. An element $x$ in a graded ring $R$ is \underline{homogeneous} if there exists some $n$ s.t. $x \in R_n$. A \underline{graded map of degree $d$} is a map between the $d$-th grade in the domain and image. 
\end{definition}

\begin{notation}
    The grade of a homogeneous element in a graded ring or a graded map is often denoted as $\abs{\cdot}$.
\end{notation}

\begin{definition}[Homogeneous Derivation]
    Given a graded algebra $A$ which has a vector space structure, and a homogeneous linear map $D$, $D$ is a \textbf{homogeneous derivation} if for all $a, b \in A$ it satisfies the following condition:
    \[
        D(ab) = D(a)b + \varepsilon^{\abs{a}\abs{D}} \cdot aD(b)
    \]
    $\varepsilon$ is the \underline{commutative factor} which is 1 if $A$ is commutative, and $-1$ if $A$ is anti-commutative. 
\end{definition}

\begin{remark}
    In the case where $\varepsilon = 1$ the derivation is simply the Leibniz rule. If $\varepsilon = -1$, and further $D$ is odd, we have
    \[
        D(ab) = D(a)b + (-1)^{\abs{a}} aD(b)
    \]
    In this case $D$ is an \underline{anti-derivation}.
\end{remark}

\begin{notation}
    The $k$-forms on a manifold $M$ ($\Lambda^k(\dual{T_p} M)$) is often denoted by $\Omega^k(M)$. These notations are used interchangeably in the followings. 
\end{notation}

\begin{definition}[Exterior Derivative]
    The \textbf{exterior derivative} is an $\R$-linear map $\d: \Lambda^k(\dual{T_p} M) \to \Lambda^{k + 1}(\dual{T_p} M)$ s.t.
    \begin{enumerate}[label=\arabic*)]
        \item For a 0-form $f$, $\d f$ is differential as real-valued functions.
        \item $\d(\d f) = 0$ for any 0-form $f$.
        \item $\d$ is an anti-derivation of degree 1 on the exterior algebra, i.e. it satisfies the following condition:
        \[
            \d(a \wedge b) = (\d a) \wedge b + (-1)^{\abs{a}} a \wedge (\d b)
        \]
    \end{enumerate}
\end{definition}

\begin{remark}
    The exterior derivative is uniquely defined. This can be seen by the fact that the exterior derivative of all $0$-forms are determined by the uniqueness of derivative of real-valued functions; and inductively the derivative of $n$-forms can be expressed as the wedge product of the differential of two lower degree derivatives. 
\end{remark}

\begin{remark}
    The exterior derivative is linear, i.e. $\d^2 = 0$ as an operator. This results from the fact that the external derivative of $n$-forms for all $n$ can be reduced to the external derivative of lower degree forms, and eventually to $0$-forms. We then can get the result from Axiom 2).
\end{remark}

\begin{remark}
    In particular, if $f$ is a $0$-form (i.e. scalar-valued functions), then for any differential form $a$,
    \[
     \d(fa) = \d(f \wedge a) = (\d f) \wedge a + f \wedge (\d a)
    \]
    as when one of the multiplicands is a $0$-form, or in general scalars, scalar multiplication and wedge product are equivalent. 
\end{remark}

\begin{example}
    Consider the dual space of $\R^n$, with covectors $\d x^i$ for $i \in \llbracket 1, n \rrbracket$ a basis. Let $I \subseteq \{1, \dots, n\}$ be the index set, and $i_1, \dots, i_k \in I$. Consider the exterior derivative of
    \[
        \varphi = f(\d x^{i_1} \wedge \dots \wedge \d x^{i_k}) =  f \wedge \d x^{i_1} \wedge \dots \wedge \d x^{i_k}
    \]
    we have by axiom 2)
    \begin{align*}
        \d \varphi
        & = \d (f \wedge \d x^{i_1} \wedge \dots \wedge \d x^{i_k}) \\
        & = \d f \wedge \d x^{i_1} \wedge \dots \wedge \d x^{i_k} + \sum_{j = 1}^{\abs{I} = k} (-1)^{j - 1} f \wedge \d x^{i_1} \wedge \dots \wedge \d^2 x^{i_p} \wedge \dots \wedge \d x^{i_k} \\
        & = \d f \wedge \d x^{i_1} \wedge \dots \wedge \d x^{i_k} \\
        & = (\pp{f}{x^i} \d x^i) \wedge (\d x^{i_1} \wedge \dots \wedge \d x^{i_k})
    \end{align*}
    where $f_i = \pp{f}{x^i}$ are the coefficients of the decomposition of $f$ into the covectors in the basis. In particular, if $\varphi$ is top-degree (i.e. $k = n$), then for all $i$ there exists some same $\d x^i$ in the following term ($\d \varphi$ becomes $(n+1)$-degree form). By anti-symmetry of the exterior algebra, this vanishes.

    The conclusion then, is that the exterior derivative of top-degree forms vanish. 
\end{example}

\textstart
The following introduces the dual notion of exterior derivative, which instead of raising the degree of the forms lowers it by 1.

\begin{definition}[Tensor]
    A \textbf{$(k, \ell)$-tensor} on a vector space $V$ is
    \[
        \Pi^{(k, \ell)}(V) = \underbrace{V \tensor \cdots \tensor V}_{k} \tensor \underbrace{\dual{V} \tensor \cdots \tensor \dual{V}}_{\ell}
    \]
\end{definition}

\textstart
A comprehensive introduction to tensors in the context of manifolds will be shown in chapter 1. The definition here is displayed only to ensure that we can have a rigorous definition of the followings:

\begin{definition}[Tensor Contraction]
    For a given vector space $V$, a \textbf{tensor contraction} is the operation which evaluates a covector on a vector:
    \begin{align*}
        C: \Pi^{(k, \ell)}(V)\quad & \to \quad \Pi^{(k - 1, \ell - 1)}(V) \\[2pt]
        (v_1 \tensor \dots \tensor v_k \tensor v_1^{\ast} \tensor \dots \tensor v_{\ell}^{\ast})\quad & \mapsto \quad v_{\ell}^{\ast}(v_k) (v_1 \tensor \dots \tensor v_{k - 1} \tensor v_1^{\ast} \tensor \dots \tensor v_{\ell - 1}^{\ast})
    \end{align*}
\end{definition}

\begin{definition}[Interior Derivative]
    An \textbf{interior derivative (interior product)} is a tensor contraction of differential forms with a vector field:
    \[
        \iota: \mathfrak{X}(M) \times \Omega^{k}(M) \to \Omega^{k-1}(M) \qquad \iota(X, \omega)(X_1, \dots, X_{n-1}) = \omega(X, X_1, \dotsm X_{n-1})
    \]
    The vector field is often written as subscripts. That is, $\iota(X, -)$ is denoted as $\iota_X(-)$.
\end{definition}

\begin{proposition}
    Interior Derivative is the unique anti-derivation of degree -1, i.e. for $a$ a $p$-form and $b$ a $q$-form, we have
    \[
        \iota_X(a \wedge b) = (\iota_X a) \wedge b + (-1)^p a \wedge (\iota_X b)
    \]
\end{proposition}

\begin{proof}
    Use the equation for computing wedge of differential forms, given in Remark \ref{rmk: wedge of differential forms}:
    \begin{align*}
        \iota_{w_1}(a \wedge b)(w_2, \dots, w_{p+q})
        & = (a \wedge b) (w_1, \dots, w_{p+q}) \\
        & = \sum_{\sigma \in S(p, q)} \operatorname{sgn}(\sigma) a(w_{\sigma(1)}, \dots, w_{\sigma(p)}) b(w_{\sigma(p+1)}, \dots, w_{\sigma(p+q)}) 
    \end{align*}
    Since by definition $\sigma \in S(p, q)$ satisfies
    \[
        \sigma(1) < \cdots < \sigma(p), \qquad \sigma(p+1) < \cdots < \sigma(p+q)
    \]
    $1$ can only be either $\sigma(1)$ or $\sigma(p+1)$ (as they are the smallest element in the corresponding set). Therefore we can do the decomposition:
    \begin{align*}
        \operatorname{LHS} 
        & = \sum_{\sigma(1) = 1, \sigma \in S(p, q)} \operatorname{sgn}(\sigma) a(w_{1}, w_{\sigma(2)}, \dots, w_{\sigma(p)}) b(w_{\sigma(p+1)}, \dots, w_{\sigma(p+q)}) \\
        & \qquad + \sum_{\sigma(p+1) = 1, \sigma \in S(p, q)} (-1)^p\operatorname{sgn}(\sigma) a(w_{\sigma(1)}, \dots, w_{\sigma(p)}) b(w_1, w_{\sigma(p+2)}, \dots, w_{\sigma(p+q)}) 
    \end{align*}
    as $\sigma(1) = 1$ does not contribute any inversion and does not change the signature of the permutation; but $\sigma(p+1) = 1$ contributes $p$ inversions, which contributes a factor $(-1)^p$ to the permutation. But see that in both expressions we have $w_1$ in the first position in the evaluation of the differential form. This gives the expression using interior product:
    \[
        \operatorname{LHS} = (\iota_{w_1} a) \wedge b + (-1)^p a \wedge (\iota_{w_1} b)
    \]
    which is exactly the desired formula. 
\end{proof}

\section{Lie Groups}

\textstart
This part provides an elementary introduction to Lie Groups and associated results, consulting \href{https://scholar.rose-hulman.edu/rhumj/vol15/iss2/5}{this note}. Some of the results may be reiterated in the main notes, and what provided here is only an elementary approach to the concepts. 

\end{document}