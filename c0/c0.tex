\documentclass{article}
\usepackage{../refalg}

\begin{document}
\Makepagesectionhead{MATH 635}{Basic Definitions and Results}{ARessegetes Stery}

\tableofcontents
\newpage

Here lists some definitions that are assumed to be familiar to students in the lecture. They are listed for an acquaintance with the basic definitions, and for lookup.

\section{Differential Form}

This part seeks to introduce the differential between tangent bundles of manifolds, and its dual operation. To formalize this, some preliminaries from algebra need to be introduced first:

\begin{definition}[Tensor Algebra]
    Let $R$ be a ring, with $M$ an $R$-module. Then the \textbf{tensor algebra} on $M$ is defined to be
    \[
        T(M) := \bigoplus_{i \geq 0} T^i(M) \qquad\text{where } T^i(M) := \underbrace{M \tensor_R \cdots \tensor_R M}_{i \text{ times}}
    \]
\end{definition}

\begin{definition}[Exterior Algebra]
    Let $T(M)$ be the tensor algebra on $M$. Then the \textbf{exterior algebra} on $M$ is defined as
    \[
        \Lambda(M) := T(M) / I , \qquad \text{where } I := \left\{ \inner{x \tensor x} \mid x \in M \right\}
    \]
    The equivalence classes in $\Lambda(M)$ is denoted by the \underline{wedge prduct}, i.e.
    \[
        a \wedge b := [a \tensor b] = (a \tensor b) \mod I
    \]
\end{definition}

\begin{remark}
    By the quotient applied to the tensor algebra, it is alternative. Specifically, for all $x, y \in M$, $x \wedge x = 0$; and $x \wedge y = -y \wedge x$.
\end{remark}

\begin{remark}
    Similar to the tensor algebra, the exterior algebra can be decomposed as
    \[
        \Lambda(M) = \bigoplus_{i \geq 0} \Lambda^i(M), \qquad \text{where } \Lambda^i(M) := T^i(M) / \{ \inner{x \tensor x} \subset T^i M \mid x \in M \}
    \]
\end{remark}

\begin{definition}[Tangent Space]
    There are two equivalent definitions of tangent spaces, via specifying its elements \textbf{tangent vectors}. Let $M$ be a smooth $n$-manifold, with $(U, \varphi)$ a local chart of it. Then the \textbf{tangent space} of $M$ at $p$ is defined as:
    \begin{itemize}
        \item \emph{Tangent vectors as equivalence class of curves.} Let $\gamma: R \supset I \to M$ be a curve on $M$. Define the equivalence class of such curves as
        \[
            \gamma_1 \sim \gamma_2 \quad \iff \quad (\varphi \circ \gamma_1)'(0) = (\varphi \circ \gamma_2)'(0)
        \]
        The \textbf{tangent vector}s are then equivalence classes of curves on $M$. 
        
        Considering the map $(\varphi \circ -)'(0) : T_p M \to \R^n$, which by the definition of the equivalence relation is a bijection. The vector space structure is then induced by the vector space structure in $\R^n$.
        \item \emph{Tangent vectors as derivation.} Denote $f: M \to \R$ as $\smooth(M)$ if $f$ is smooth, i.e. for chart of $M$ $(U, \varphi)$ and $p \in U$, then in the neighborhood of $p$, $f \circ \varphi$ is smooth as maps from $\R^n$ to itself. Then a \textbf{tangent vector} is a map $\partial: \smooth(M) \to \R$ s.t.
        \[
            \forall f, g \in \smooth(M), \qquad \partial (f \cdot g) = \partial(f) \cdot g + f \cdot \partial(g)
        \]
        i.e. the \underline{Leibniz Identity}. The multiplication $\cdot$ is the multiplication in $R$ after applying the map. The vector spae structure is specified via requiring the functions to be linear, i.e. define
        \[
            \partial(\lambda f + \mu g) = \lambda \cdot \partial(f) + \mu \cdot \partial(g), \qquad \forall f, g \in \smooth(M), \lambda,\ \mu \in \R
        \]
    \end{itemize}
\end{definition}

\begin{definition}[Cotangent Space]
    The \textbf{cotangent space}, as the dual of tangent space, is defined similarly in two equivalent ways:
    \begin{itemize}
        \item \emph{Explicitly as dual of tangent space.} For $p \in M$ which is a smooth $n$-manifold, the \textbf{cotangent vectors} $\dual{T_p} M := \dual{(T_p M)}$ are \emph{linear} functions $T_p M \to \R$ (under the context of real manifolds). The cotangent space is generated by the cotangent vectors, with the vector space structure induced by the real vector space.
        \item \emph{Equivalence class of smooth functions.} Let $f, g \in \smooth(M)$. Then for $p \in M$, 
        \[
            f \sim g \quad \iff \quad \restr{D(f - g)}{p} = 0
        \]
        The cotangent space consists of equivalence class of such smooth functions. The vector space structure is the same as that of real functions.
    \end{itemize}
\end{definition}

\begin{remark}
    In the second definition for cotangent space, this is exactly the dual of the second definition of tangent spaces. This can be seen by the fact that if two functions are in the same equivalence, the derivation on them should be identical.
\end{remark}

\begin{definition}[Differential Form]
    Let $M$ be a manifold, with $U$ an open subset of $M$. Then a \textbf{$\bm{k}$-form} on $U$ is a map which associates every $p \in U$ an element in the exterior algebra $\omega_p \in \Lambda^k(\dual{T_p} M)$
\end{definition}

\begin{remark}
    In particular, cotangent vectors are $1$-forms.
\end{remark}

\begin{remark}
    By the universal property of tensor product, for $\wedge^k(M)$ we have the isomorphism
    \[
        \dual{(w_1 \wedge \dots \wedge w_k)} \simeq \dual{w_1} \wedge \dots \dual{w_k}
    \]
    i.e. there is an induced isomorphism between tensor of $1$-forms and $k$-forms, via taking the tensor product of the $1$-forms.
\end{remark}

\begin{definition}[Pushforward, Pullback]
    Let $F: M \to N$ be a smooth map between manifolds, and $p \in M$. Then
    \begin{itemize}
        \item The \textbf{pushforward (differential)} of $F$ at $p$ is defined as
        \[
            \d_p F: T_p M \to T_{F(p)} N, \qquad \d_p F (\partial) (\varphi) = 
        \]
        for $\partial \in T_p M \simeq (\smooth(M) \to \R), \varphi \in \smooth(M)$.
        \item The \textbf{pullback} of $F$ at $p$ is defined on the \underline{covector field}, or \underline{linear forms}, defined as 
        \[
            \dual{(\d_p F)}: \dual{T_{F(p)}}N \to \dual{T_p} M, \qquad \dual{(\d_p F)}(f)(\partial) = f(\d_p F(\partial))
        \]
        where $f \in \dual{T_{F(p)}} N, \partial \in T_p M$.
    \end{itemize}
\end{definition}

\begin{definition}[Immersion, Submersion]
    The following is a pair of dual notions which specifies the differential pushforward of a smooth map $f: M \to N$:
    \begin{itemize}
        \item $f$ is an \textbf{immersion} if for all $p \in M$, $d_p f : T_p  M \to T_{f(p)} N$ is injective.
        \item $f$ is a \textbf{submersion} if for all $p \in M$, $d_p f : T_p  M \to T_{f(p)} N$ is surjective.
    \end{itemize}
\end{definition}

\end{document}