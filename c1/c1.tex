\documentclass{article}
\usepackage{../refalg}

\begin{document}
\Makepagesectionhead{MATH 635}{Basic Riemannian Constructions}{ARessegetes Stery}

\tableofcontents
\newpage

\section{Riemannian Structure}

\begin{definition}[Riemannian Structure]
    Let $M$ be a smooth $n$-manifold. Then a \textbf{Riemannian Structure} on it is an assignment $M \ni p \mapsto g_p$, where $g_p: T_p M \times T_p M \to \R$ is a bilinear, positive-definite, symmetric form, that depends smoothly on $p$. Such $g$ is a \textbf{Riemannian metric}. 
    
    Specifically, if $(x^1, \dots, x^n)$ is a coordinate system on $U \subseteq M$, then for $i, j \in \llbracket 1, n \rrbracket$, $p \in U$, define
    \[
        g_{ij}(p) = g_p\left(\restr{\pp{}{x^i}}{p}, \restr{\pp{}{x^j}}{p}\right)
    \]
    where $\restr{\pp{}{x^i}}{p} \in T_p M$ for all $i$. Then $g_{ij}$ is $\smooth$; and $g(p) = (g_{ij}(p))$ is a symmetric matrix that depends on $p$. The matrix is often referred to as the \textbf{metric tensor}. Evaluation on the metric can be done via $g_p(v_1, v_2) = v_1^T g(p) v_2$.
\end{definition}

\begin{example}
    Let $M \subseteq \R^N$ be a smooth manifold. Then for all $p \in M$, via embedding the tangent space into $\R^N$, $T_p M \subseteq \R^N$. The inner product in the usual sense (dot product in $\R^n$) gives $M$ a Riemannian structure. This implies that Euclidean Space obtains a Riemannian structure. 
\end{example}

\begin{example}
    Take $M = S^2 \subset \R^3$, and let $U = S^2 \cap \{ (x_1, x_2, x_3) \in \R^3 \mid x_3 > 0 \}$. Specify the Riemannian structure as the inner product in $\R^3$, with the tangent space regarded as planes in $\R^3$, taking the local coordinate system as $(x_1, x_2)$, then at $(x_1, x_2)$, the metric tensor for the tangent space is given by
    \[
        g = \frac{1}{1- (x_1^2 + x_2^2)} \begin{pmatrix}
            1 - x_2^2 & x_1 x_2 \\
            x_1 x_2 & 1 - x_1^2
        \end{pmatrix}
    \]
    To see that this is indeed the metric, notice that at $T_{(x_1, x_2)} M$, the normal vector is given by $(x_1, x_2, \sqrt{1 - x_1^2 - x_2^2})$. Therefore $\bm{\alpha} \in T_{(x_1, x_2)}M$ must be in the form of $(a, b, -\frac{a x_1 + b x_2}{\sqrt{1 - x_1^2 - x_2^2}})$ for $a, b \in \R$. Let $\bm{\beta} := (c, d, -\frac{c x_1 + d x_2}{\sqrt{1 - x_1^2 - x_2^2}}) \in T_{(x_1, x_2)} M$. Then
    \begin{align*}
        \inner{\bm{\alpha}, \bm{\beta}} 
        & = ac + bd + \frac{ac x_1^2 + bd x^2 + (ad + bc)(x_1 x_2)}{1 - x_1^2 - x_2^2} \\
        & = \frac{1}{1 - x_1^2 - x_2^2} \left( ac(1 - x_2^2) + bd(1 - x_1^2) + ad(x_1 x_2) + bc(x_2 x_1) \right)
    \end{align*}
    where the entries of the metric tensor can be read off. 
\end{example}

\begin{observation}
    The \underline{length} can thus be defined given the generalization of inner product on the structure. For $\gamma: [0, 1] \to M$, the length of $\gamma$
    \[
        \norm{\gamma} = \int_{0}^{1} \sqrt{g_{\gamma(t)} \left(\dd{}{t}{\gamma(t)}, \dd{}{t}{\gamma(t)}\right)}\d t =: \int_{0}^{1} \sqrt{g_{\gamma(t)} \left(\dot{\gamma}, \dot{\gamma}\right)} \d t
    \]
    If $M$ is connected, Then the distance between $a, b \in M$ is $\inf\limits_{\gamma(0) = a, \gamma(1) = b} \norm{\gamma}$.
\end{observation}

\section{Vector Bundle}

\begin{definition}[Vector Bundle]
    A \textbf{(real) vector bundle of rank $k$ over $M$} is a surjection $\pi: \mathcal{E} \to M$ satisfying the following properties:
    \begin{enumerate}
        \item For all $p \in M$, $\mathcal{E}_p := \pi^{-1}(p)$ has a real vector space structure of dimension $k$.
        \item For all $p \in M$, there exists an open neighborhood $U \subset M$ containing $p$, and a diffeomorphism $\chi: \pi^{-1}(U) \iso U \times \R^n$ together with the projection $p: U \times \R^n \to U$ s.t.
        \begin{itemize}
            \item $\restr{\pi}{U} = p \circ \chi$, i.e. the following diagram commutes:
            
            \begin{minipage}{\linewidth}
                \centering
                \begin{tikzcd}
                    \pi^{-1}(U) \arrow[rr, "\chi"] \arrow[rrdd, "\restr{\pi}{\pi^{-1}(U)}"] & & U \times \R^n \arrow[dd, "p"] \\
                    & & \\
                    & & U
                \end{tikzcd}
            \end{minipage}
            \item For all $p \in U$, $\restr{\chi}{p}: \mathcal{E}_p \to \{p\} \times \R^n$ is a linear isomorphism.
        \end{itemize}
        $E$ is the \underline{total space}, and $M$ is the \underline{base}. The map $\chi$ is called the \underline{local trivialization} of $E$ at $p$.
    \end{enumerate}
\end{definition}

\begin{example}
    The following gives some examples of vector bundle:
    \begin{enumerate}
        \item The vector bundle which associates every point in $M$ the vector space $\R^n$, given by $\pi: M \times \R^n \to M$ is the trivial bundle. 
        \item Consider $\mathcal{E} = TM$ which is the tangent bundle (or isomorphically, the cotangent bundle $\dual{T}M$), defined as $TM = \coprod_{p \in M} T_p M$. Then if $\phi = (x^1, \dots, x^n)$ is a coordinate system on $U \subset M$, we have a basis $\left\{ \pp{}{x^1}, \dots, \pp{}{x^n} \right\}$ a basis of $T_p M$ for each $p \in M$. The map $\chi$ is then given by 
        \[
            \chi: \pi^{-1}(U) = TU \to U \times \R^n, \quad (p, T_p M \ni v) \mapsto (p, \inner{v^1, \dots, v^n}) \text{ s.t. } v = \sum_{i = 1}^{n} v^i \pp{}{x^i}
        \]
        $\chi$ is a diffeomorphism as the tangent space $TM$ has a smooth structure. Similarly one can consider the dual $\dual{T}M$ with basis $\{\d x^1, \dots, \d x^n\}$. Then the corresponding map becomes
        \[
            \chi: \pi^{-1}(U) = \dual{T}U \to U \times \R^n, \quad (p, \dual{T_p}M \ni \alpha) \mapsto (p, \inner{a_1, \dots, a_n}) \text{ s.t. } \alpha = \sum_{i = 1}^n a_i\d x^i
        \]
        where $\alpha: T_p M \to \R$ is a linear map. 
        \item Let $M \subset \R^N$ be a $n$-manifold. Define $\mathcal{E} = \left\{ (p, v) \mid p \in \mathcal{E}, v \in (T_pM)^{\perp} \right\}$, The map can be defined as
        \[
            \chi: \pi^{-1}(U) \to U \times \R^{N - n}, \quad (p, v) \mapsto (p, v)
        \]
        which is the identity if one identifies $T_p M$ with a subspace of $\R^N$.
    \end{enumerate}
\end{example}

\begin{definition}[Section]
    Let $\pi: \mathcal{E} \to M$ be a vector bundle. Then a \textbf{(smooth) section} is a (smooth) map $s: M \to \mathcal{E}$ s.t. $\pi \circ s = \Id_M$. The set of all smooth sections is denoted as $\Gamma(\mathcal{E}, M)$, or simply $\Gamma(\mathcal{E})$.
\end{definition}

\begin{remark}
    Explained in plain words, a section selects an element $s(p) \in \mathcal{E}_p$ for each $p \in M$. Let $\mathcal{E} = TM$, and identifying the tangent space with $\R^N$ for some $N$, then a section $s$ gives a vector field on $M$.
\end{remark}

\begin{remark}
    $\Gamma(\mathcal{E})$ defines a module structure over $\smooth(M)$, with the scalar multiplication defined as $\chi(f \cdot s(x)) = (x, f(x) \cdot (p_r\circ \chi \circ s)(x))$, where $p_r: (U, \R^n) \to \R^n$ is the projection that takes the second field. This map is smooth, as both $f, p_r, \chi$ and $s$ are smooth. 
\end{remark}

\begin{definition}[Moving Frame]
    Let $\mathcal{E} \to M$ be a vector bundle, and $U \subset M$ an open set. Then a \textbf{moving frame} of $\mathcal{E}$ on $U$ is an $r$-tuple $(E_1, \dots, E_r)$ s.t. for all $j$, $E_j$ is a section of $\mathcal{E} \supset \pi^{-1}(U) \to U$; and for all $p \in U$, $(E_1(p), \dots, E_r(p))$ gives a basis of $\mathcal{E}_p$.
\end{definition}

\begin{remark}
    There is a bijection between moving frames of $\mathcal{E}$ on $U$ and trivializations $\chi: \pi^{-1}(U) \to U \times \R^n$:
    \begin{itemize}
        \item Given a moving frame $(E_1, \dots, E_r)$ of $\mathcal{E}$ on $U$, define
        \[
            \chi: \pi^{-1}(U) \to U \times \R^n, \quad v \mapsto (\pi(v), \inner{v^1, \dots, v^n}) \text{ s.t. } v = \sum_{i = 1}^n v^i E_i(p)
        \]
        \item Conversely, given a trivialization $\chi: \pi^{-1}(U) \to U \times \R^n$, define
        \[
            E_i: M \supset U \to \pi^{-1}U \subset \mathcal{E}, \quad p \mapsto \chi^{-1}(p, \langle 0, \underset{i\text{-th position}}{\underset{\uparrow}{\dots, 0, 1, 0, \dots}}, n\rangle)
        \]
        The vector space structure is induced by the vector space structure of $\R^n$.
    \end{itemize}
\end{remark}

\begin{definition}[Transition]
    Let $\pi: \mathcal{E} \to M$ be a vector bundle, and $U_{\alpha}, U_{\beta} \subset M$ two open subsets of $M$, and $\chi_{\alpha}, \chi_{\beta}$ be the corresponding local trivializations. Define $\phi_{\alpha, p} := p_r \circ \chi_{\alpha}$ where $p_r: (U, \R^n) \to \R^n$ is the projection. Then the \textbf{transition} between $\chi_{\alpha}$ and $\chi_\beta$ is a map $\tau_{\alpha \beta}$ s.t. for $f_{\alpha\beta}: U \times \R^n \to U \times \R^n, (p, v) \mapsto (p, \tau_{\alpha \beta}(v))$, the following diagram commutes:

    \begin{minipage}{\linewidth}
        \centering
        \begin{tikzcd}
            \pi^{-1}(U_{\alpha} \cap U_{\beta}) \arrow[rr, "\chi_{\alpha}"] \arrow[rrdd, "\chi_{\beta}"] & & (U_{\alpha} \cap U_{\beta}) \times \R^r \arrow[dd, "f_{\alpha\beta}"] \\
            & & \\
            & & (U_{\alpha} \cap U_{\beta}) \times \R^r
        \end{tikzcd}
    \end{minipage}
\end{definition}

\begin{observation}
    $\tau_{\alpha \beta}: U_{\alpha} \cap U_{\beta} \to \GL_r(\R)$. It is smooth as $\chi$ is a diffeomorphism.
\end{observation}

\begin{remark}
    By definition it is clear that $\tau_{\alpha \beta} = \tau_{\beta \alpha}^{-1}$; and $\tau_{\alpha \beta} \circ \tau_{\beta \gamma} = \tau_{\alpha \gamma}$.
\end{remark}

\begin{observation}
    By the fact that any object defined on the vector space can be transferred to the vector bundle via the local trivialization since it is a diffeomorphism, constructions on vector spaces can be identified with constructions on vector bundles. 
\end{observation}

\begin{lemma}[Bundle Chart Lemma]
    Given $M$ a smooth manifold, and for all $p \in M$ associate an $r$-dimensional vector space $\mathcal{E}_p$ to it. Define $\mathcal{E} := \coprod_{p \in M} \mathcal{E}_p$, and $\pi: \mathcal{E} \to M, (p, v) \mapsto p$. Further the followings are satisfied:
    \begin{itemize}
        \item There exists an open cover $\{U_{\alpha}\}$ of $M$.
        \item For all $\alpha$, there exists a bijective map $\chi_{\alpha}: \pi^{-1}(U_{\alpha}) \iso U_{\alpha} \times \R^n$ s.t. $\pi = p \circ \chi_{\alpha}$ where $p$ is the projection $U_{\alpha} \times \R^n \to U_{\alpha}$.
    \end{itemize}
    Then there exists a unique $\smooth$ manifold structure on $\mathcal{E}$ s.t. $\chi_{\alpha}$ is a diffeomorphism; and $\mathcal{E}$ is a $\smooth$ bundle over $M$. 
\end{lemma}

\textstart
The proof of the lemma is via verifying that $\mathcal{E}$ is indeed a manifold, and is omitted here. The more interesting aspect is that this provides a lot of operations on bundles, for example combination of two bundles:

\begin{example}\label{ex: direct sum of vector bundle}

    Let $(U_{\alpha})$ be an open cover of $M$, and $\pi': \mathcal{E}' \to M$, $\pi'': \mathcal{E}'' \to M$ be two vector bundles over $M$. Define $\mathcal{E}' \oplus \mathcal{E}'' \to M$, where $\mathcal{E}' \oplus \mathcal{E}'' = \coprod_{p \in M} (\mathcal{E}_p' \oplus \mathcal{E}_p'')$, Let $(U_{\alpha})$ be a cover of $M$ s.t. both $\mathcal{E}'$ and $\mathcal{E}''$ trivialize over each $U_{\alpha}$. Then it is valid to define the trivialization $\chi: \pi^{-1}(U_{\alpha}) \to U_{\alpha} \times (\R^{r'} \oplus \R^{r''})$ s.t. $\phi_{\alpha, p} = \phi_{\alpha, p}' \oplus \phi_{\alpha, p}''$. In matrices this gives a block diagonal matrix. 
\end{example}

\section{Tensors}

\begin{definition}[Free Vector Space]
    Let $S$ be a (possibly infinite) set, and $K$ a field. The \textbf{free vector space} generated by $S$ over $K$ is defined as the set
    \[
        FS := \left\{ \sum_{s \in S} \alpha_s s \bigmid \text{finite nonzero } \alpha_s \right\}
    \]
    satisfying the axioms of vector spaces, i.e. linear in $s$ with addition and scalar multiplication.
\end{definition}

\begin{definition}[Tensor Product]
    Let $V, W$ be finite dimensional vector space. The \textbf{tensor product of $V$ and $W$}, denoted $V \tensor W$ is the free vector space generated by $(V \times W)/ \sim$, where $\sim$ follows the rules:
    \begin{itemize}
        \item $(v + v', w) \sim (v, w) + (v', w)$ for $v, v' \in V, w \in W$.
        \item $(v, w + w') \sim (v, w) + (v, w')$ for $v \in V, w, w' \in W$.
        \item $(cv, w) \sim c(v, w) \sim (v, cw)$ for $c \in K, v \in V, w \in W$.
    \end{itemize}
    This gives a map $\tensor: V \times W \to V \tensor W$, $(v, w) \mapsto [(v, w)]$ which is an equivalence class in $V \tensor W$.
\end{definition}

\begin{proposition}[Universal Property of Tensor Product]
    Let $Z$ be a real vector space, and the map $b: V \times W \to Z$ bilinear. Then there exists a unique linear map $f: V \tensor W \to Z$ s.t. the following diagram commute:

    \begin{minipage}{\linewidth}
        \centering
        \begin{tikzcd}
            V \times W \arrow[rrdd, "b"] \arrow[rr, "\tensor"] & & V \tensor W \arrow[dd, "f"] \\
            & & \\
            & & Z \\
        \end{tikzcd}
    \end{minipage}
\end{proposition}

\begin{corollary}
    Let $(v_i)$ be a basis of $V$, and $(w_i)$ a basis of $W$. Then $(v_i \tensor w_j)$ gives a basis of $V \tensor W$, with $\dim V \tensor W = \dim V \cdot \dim W$.
\end{corollary}

\begin{remark}
    Not every element in $V \tensor W$ can be expressed as $v \tensor w$ for some $v \in V, w \in W$. In particular, one can have $v' = \sum_{i \in I} v_i \tensor w_i \in V \tensor W$, whose representative cannot be further simplified. 
\end{remark}

\begin{proposition}
    Let $V$ and $W$ be finite dimensional real vector space. Then there exists a natural isomorphism $V \tensor W \iso \Hom(\dual{V}, W)$. In particular, this gives $\dual{V} \tensor \dual{V} \simeq \Hom(V, \dual{V}) \simeq \{\text{bilinear forms } V \times V \to \R\}$.
\end{proposition}

\begin{proof}
    To construct the map it suffices to give a bilinear map $V \times W \to \Hom(\dual{V}, W)$. Define it as
    \[
        (v, w) \mapsto (\alpha \mapsto \alpha(v) w) \in \Hom(\dual{V}, W)
    \]
    Applying the universal property gives the desired result. 
\end{proof}

\begin{remark}\label{rmk: tensor isomorphic to multilinear maps}
    This can be extended to multiple tensor products: 
    \[
        \underbrace{V \tensor \cdots \tensor V}_{k} \tensor \underbrace{\dual{V} \tensor \cdots \tensor \dual{V}}_{\ell} \quad \simeq \quad (\underbrace{\dual{V} \times \cdots \times \dual{V}}_{k} \times \underbrace{V \times \cdots \times V}_{\ell} \to \R) 
    \]
    where RHS is a multilinear map. 
\end{remark}

\begin{definition}[Tensor]
    Denote
    \[
        G = \underbrace{V \tensor \cdots \tensor V}_{k} \tensor \underbrace{\dual{V} \tensor \cdots \tensor \dual{V}}_{\ell}
    \]
    Then an element in $G$ is called a \textbf{$(k, \ell)$-tensor} on $V$. $k$ is the \underline{contravariant degree}, and $\ell$ is the \underline{covariant degree}.
\end{definition}

\begin{remark}
    It is possible to define the tensor product of maps. Namely, for $f: V \to V'$ and $g: W \to W'$, there exists a unique linear map $f \tensor g: V \tensor W \to V' \tensor W'$ s.t. $(f \tensor g) (v \tensor w) = (f(v)) \tensor (g(w))$. This can be seen via consider first the $\R$-balanced linear map $V \times W \to V \tensor W$; and apply the universal property of tensor product gives the desired result. 
\end{remark}

\begin{observation}
    Similar to Example \ref{ex: direct sum of vector bundle} where we considered the direct sum of vector bundles as a new bundle, we can consider the tensor product of vector bundles. For $\mathcal{E}'$ and $\mathcal{E}''$ two vector bundles, $\coprod_{p \in M} \mathcal{E}_p' \tensor \mathcal{E}_p''$ gives a vector bundle structure over $M$. In particular, this can be extended to the tangent and cotangent spaces of a manifold.
\end{observation}

\begin{notation}
    Given a smooth manifold $M$, for $k, \ell \in \Z_{\geq 0}$, denote
    \[
        \mathscr{T}^{(k, \ell)} M = \Pi^{(k, \ell)} M = \underbrace{TM \tensor \cdots \tensor TM}_{k} \tensor \underbrace{T^{\ast}M \tensor \cdots \tensor T^{\ast} M}_{\ell}
    \]
    In particular, this is $TM$ for $k = 1, \ell = 0$, and is $T^{\ast}M$ for $k = 0, \ell = 1$. By identification in Remark \ref{rmk: tensor isomorphic to multilinear maps} maps $\Pi^{(k, \ell)} M \to M$ are always multilinear (with the identification with the Euclidean Space). 
\end{notation}

\begin{definition}[Tensor Field]
    Let $M$ be a smooth manifold. A \textbf{$(k, \ell)$-tensor (field) on $M$} is a smooth section of $\Pi^{(k, \ell)} M$.
\end{definition}

\textstart
In local coordinates, if we have local coordinates $(x^1, \dots, x^n)$ on $U \subseteq M$, then any element in $\Pi^{(k, \ell)} M$ is in the form of 
\[
    f_{j_1, \dots, j_{\ell}}^{i_1, \dots, i_k} := \ppd{x^{i_1}} \tensor \cdots \tensor \ppd{x^{i_k}} \tensor \d x^{j_1} \tensor \cdots \tensor \d x^{j \ell}
\]
where $i_k$s and $j_{\ell}$s run over $\{ 1, \dots, n \}$ arbitrarily. A section on $\Pi^{(k, \ell)} M$ is then $\sum_{i, j} f_{j_1, \dots, j_{\ell}}^{i_1, \dots, i_k}$, with it being smooth implies that all such $f$s are smooth (as functions).

\section{Riemannian Metric}

\begin{definition}[Riemannian Metric]
    Given a Riemannian manifold $M$, a \textbf{Riemannian Metric} on $M$ is a (0, 2)-tensor $g$ that is symmetric and positive definite. In local coordinates,
    \[
        g = \sum_{i, j = 1}^n g_{ij} \d x^i \tensor \d x^j
    \]
    That is, for all $p \in M$, $g_p: T_p M \times T_p M \to \R$ is bilinaer and symmetric; and for all $v \in T_p M$, $g_p (v, v) \geq 0$, with equality reached if and only if $v = 0$.
\end{definition}

\begin{observation}
    Notice that in local coordinates we have $v = v^i \ppd{x^i}$, $w = w^j \ppd{x^j}$. Evaluating the metric gives
    \[
        (\d x^i \tensor \d x^j) (v, w) = v^i w^j = v^j w^i = (\d x^i \tensor \d x^j) (w, v) \implies g_{ij} = g_{ji}
    \]
    Therefore, we can write
    \begin{align*}
        g 
        & = \frac{1}{2} \left( g_{ij} \d x^i \tensor \d x^j + g_{ji} \d x^j \tensor \d x^i \right)
        = \frac{1}{2} g_{ij} \left( \d x^i \tensor \d x^j + \d x^j \tensor \d x^i \right) \\
        & = g_{ij} \left( \frac{1}{2} \left( \d x^i \tensor \d x^j + \d x^j \tensor \d x^i \right) \right)
    \end{align*}
    where $\left( \d x^i \tensor \d x^j + \d x^j \tensor \d x^i \right) \in \operatorname{Sym}(T_p M)$ which is in the symmetric tensor, i.e. invariant w.r.t permutation of the entries.  
\end{observation}

\begin{definition}[Riemannian Manifold]
    A \textbf{Riemannian Manifold} is a pair $(M, g)$ where $g$ is a Riemannian metric on $M$. 
\end{definition}

\begin{definition}[Isometry]
    A \smooth-map $F: (M_1, g_1) \to (M_2, g_2)$ is an \textbf{isometry} if and only if 
    \begin{enumerate}[label=\arabic*)]
        \item $F$ is a diffeomorphism.
        \item For all $p \in M_1$, $u, v \in T_p M_1$, $g_2 (\d_p F(u), \d_p F(v)) = g_1(u, v)$.
    \end{enumerate}
    $F$ is a \textbf{local isometry} if only 2) holds. 
\end{definition}

\begin{remark}
    Since $g$ is nondegenerate, $\d_p F$ is bijective as otherwise LHS of the equality in 2) will vanish. This then implies that $\dim M_1 = \dim M_2$, otherwise $\d_p F$ will either be not injective ($\dim M_1 > \dim M_2$) or not surjective ($\dim M_1 < \dim M_2$).
\end{remark}

\begin{theorem}
    Any \smooth-manifold $M$ has a Riemannian metric.
\end{theorem}

\begin{proof}
    Let $\{ (U_{\alpha}, \varphi_{\alpha}) \}$ be an atlas of $M$. By definition of the chart we have the embedding $\varphi_{\alpha} (U_{\alpha}) \longhookrightarrow \R^n$. Define in $U_{\alpha}$ the metric s.t. $\varphi_{\alpha}$ is an isometry. In local coordinates, the metric can be expressed as $g_{\alpha} = \sum_{i} \d x_{\alpha}^i \tensor \d x_{\alpha}^j$.

    Let $(\chi_{\alpha})$ be a partition of unit subordinated to $(U_{\alpha})$ and $g = \sum_{\alpha} \chi_{\alpha} g_{\alpha}$. Nondegeneracy of the metric $g$ follows from the positivity of metric $g_{\alpha}$ and $\chi_{\alpha}$.
\end{proof}

\begin{example}
    The following gives two constructions of Riemannian Metrics:
    \begin{enumerate}[label=\arabic*)]
        \item Let $(M, g)$ be a Riemannian manifold, with $N \subset M$ a submanifold. Then $N$ inherits a Riemannian metric from the Riemannian structure on $M$, as for all $p \in N$, $T_p N \subset T_p M$.
        \item Consider Cartesian product of Riemannian manifolds. Given two Riemannian manifolds $(M_1, g_1)$, $(M_2, g_2)$, consider $M_1 \times M_2$. Then for all $(p_1, p_2) \in M_1 \times M_2$ we have the identification between the tangent spaces 
        \[
            T_{(p_1, p_2)} (M_1 \times M_2) \simeq T_{p_1} M_1 \oplus T_{p_2} M_2
        \]
        Then in coordinates, the combined metric is block diagonal: 
        \[
            g = \left(\begin{array}{@{}c|c@{}}
                g_1 & 0 \\\hline
                0 & g_2 \\
                \end{array}\right)
        \]
    \end{enumerate}
\end{example}

\section{Coverings}

\begin{definition}[Covering]
    A \smooth\ surjective map $\pi: M \to N$ is a \smooth-\textbf{covering} if for all $q \in N$, there exists $U$, a connected neighborhood of $q$ s.t. if $\pi^{-1}(U) = \amalg_{j} V_j$, where $V_j$ are connected components of $\pi^{-1}(U)$; and for all $j$, $\restr{\pi}{V_j}: V_j \to U$ is a diffeomorphism.
\end{definition}

\begin{definition}[Automorphism of Covering]
    The automorphism on the covering is defined as
    \[
        \Aut(\pi) := \{ F: M \to M \text{ diffeomorphisms} \mid \pi \circ F = \pi \}
    \]
    For any $f \in \Aut(\pi)$, it shuffles the fibers $\{ \pi^{-1}(q) \mid q \in N \}$.
\end{definition}

\begin{definition}[Normal Covering]
    A covering $\pi$ is \textbf{normal} if $\Aut(\pi)$ acts transitively on fibers. 
\end{definition}

\begin{proposition}
    Suppose that we have a manifold $M$ has a Riemannian metric, $\pi: M \to N$ is a covering. If for all $f \in \Aut(\pi)$ it is an isometry, then there exists a unique Riemannian metric on $N$ s.t. $\pi$ is a local isometry.
\end{proposition}

\begin{proof}
    Fix $q \in N$. Choose $p \in \pi^{-1}(q)$. Since $\pi$ restricted to a neighborhood of $p$ is a diffeomorphism, the differential $\d \pi_p: T_p M \to T_q N$ is bijective. Define the metric on $T_qN$ s.t. $d \pi_p$ is a linear isometry. 
    
    Check that this is compatible with the covering: if we have another choice $p' \in \pi^{-1} (q)$, by assumption there exists $F \in \Aut(\pi)$ an isometry s.t. $F(p) = F(p')$; and $\d F_p: T_p M \to T_{p'} M$ is a linear isometry. Then the composition is also a linear isometry:

    \begin{minipage}{\linewidth}
        \centering
        \begin{tikzcd}
            T_p M \arrow[rr, "\d F_p"]  & & T_{p'} M \arrow[ld, "\d \pi_{p'}"] \\
            & T_p N \arrow[lu, <-, "\d \pi_p"] &
        \end{tikzcd}
    \end{minipage}
\end{proof}

\begin{example}
    The following gives some simple examples of coverings:
    \begin{itemize}
        \item The covering $\pi: S^n \to \RP^n$, where the antipodal points are identified. $\Aut(\pi)$ is the antipodal map.
        \item Let $\Lambda \subset \R^n$ be a lattice (e.g. $\Lambda = \Z^n$). Considering $\R^n \to \R^n /\Lambda$ gives a covering map. $\Aut(\pi)$ are translations by $\Lambda$.
        \item The hyperbolic space. Let $M$ be the upper half plane $\H = \{ (x, y) \mid y > 0 \}$ with metric $\d s^2 = \frac{1}{y^2} (\d x^2 + \d y^2)$. It turns out that any compact surface of genus $> 1$ is covered by $\H$. The covering $\pi: \H \to S$ is such that $\Aut(\pi)$ acts by isometries; and the metrics on $S$ is locally isometric to the hyperbolic metric.
    \end{itemize}
\end{example}

\section{Metric on Lie Group}

\section{Common Objects on Riemannian Manifolds}

\end{document}