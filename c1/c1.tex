\documentclass{article}
\usepackage{../refalg}

\begin{document}
\Makepagesectionhead{MATH 635}{Differential Topology Preliminaries}{ARessegetes Stery}

\tableofcontents
\newpage

\section{Riemannian Structure}

\begin{definition}[Riemannian Structure]
    Let $M$ be a smooth $n$-manifold. Then a \textbf{Riemannian Structure} on it is an assignment $M \ni p \mapsto g_p$, where $g_p: T_p M \times T_p M \to \R$ is a bilinear positive definite symmetric form, that depends smoothly on $p$. Such $g$ is a \textbf{Riemannian metric}. 
    
    Specifically, if $(x^1, \dots, x^n)$ is a coordinate system on $U \subseteq M$, then for $i, j \in \llbracket 1, n \rrbracket$, $p \in U$, define
    \[
        g_{ij}(p) = g_p\left(\restr{\pp{}{x^i}}{p}, \restr{\pp{}{x^j}}{p}\right)
    \]
    where $\restr{\pp{}{x^i}}{p} \in M$ for all $i$. Then $g_{ij}$ is $C^{\infty}$; and $g(p) = (g_{ij}(p))$ is a symmetric matrix that depends on $p$. The matrix is often referred to as the \textbf{metric tensor}.
\end{definition}

\begin{definition}[Riemannian Manifold]
    A \textbf{Riemannian Manifold} is a smooth manifold $M$ endowed with a Riemannian Metric $g$, often denoted as pair $(M, g)$. 
\end{definition}

\begin{example}
    Let $M \subseteq \R^N$ be a smooth manifold. Then for all $p \in M$, via embedding the tangent space into $\R^N$, $T_p M \subseteq \R^N$. The inner product in the usual sense (dot product in $\R^n$) gives $M$ a Riemannian structure.
\end{example}

\begin{example}
    Take $M = S^2 \subset \R^3$, and let $U = S^2 \cap \{ (x_1, x_2, x_3) \in \R^3 \mid x_3 > 0 \}$. Specify the Riemannian structure as the inner product in $\R^3$, with the tangent space regarded as planes in $\R^3$, taking the local coordinate system as $(x_1, x_2)$, then at $(x_1, x_2)$, the metric tensor for the tangent space is given by
    \[
        g = \frac{1}{1- (x_1^2 + x_2^2)} \begin{pmatrix}
            1 - x_2^2 & x_1 x_2 \\
            x_1 x_2 & 1 - x_1^2
        \end{pmatrix}
    \]
    To see that this is indeed the metric, notice that at $T_{(x_1, x_2)} M$ for $(x_1, x_2) \in U$, the normal vector is given by $(x_1, x_2, \sqrt{1 - x_1^2 - x_2^2})$. Therefore $\bm{\alpha} \in T_{(x_1, x_2)}M$ must be in the form of $(a, b, -\frac{a x_1 + b x_2}{\sqrt{1 - x_1^2 - x_2^2}})$ for $a, b \in \R$. Let $\bm{\beta} := (c, d, -\frac{c x_1 + d x_2}{\sqrt{1 - x_1^2 - x_2^2}}) \in T_{(x_1, x_2)} M$. Then
    \begin{align*}
        \inner{\bm{\alpha}, \bm{\beta}} 
        & = ac + bd + \frac{ac x_1^2 + bd x^2 + (ad + bc)(x_1 x_2)}{1 - x_1^2 - x_2^2} \\
        & = \frac{1}{1 - x_1^2 - x_2^2} \left( ac(1 - x_2^2) + bd(1 - x_1^2) + ad(x_1 x_2) + bc(x_1 x_2) \right)
    \end{align*}
    where the entries of the metric tensor can be read off. 
\end{example}

\begin{observation}
    The \underline{length} can thus be defined given the generalization of inner product on the structure. For $\gamma: [0, 1] \to M$, the length of $\gamma$
    \[
        \norm{\gamma} = \int_{0}^{1} \sqrt{g_{\gamma(t)} \left(\dd{}{t}{\gamma(t)}, \dd{}{t}{\gamma(t)}\right)} =: \int_{0}^{1} \sqrt{g_{\gamma(t)} \left(\dot{\gamma}, \dot{\gamma}\right)} 
    \]
    If $M$ is connected, Then the distance between $a, b \in M$ is $\inf\limits_{\gamma(0) = a, \gamma(1) = b} \norm{\gamma}$.
\end{observation}

\section{Vector Bundle}

\section{Tensor Product}

\section{Riemannian Metric}

\section{Coverings}

\section{Metric on Lee Group}

\section{Common Objects on Riemannian Manifolds}

\end{document}